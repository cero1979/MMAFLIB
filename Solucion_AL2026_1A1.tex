% ═══════════════════════════════════════════════════════════════════════════
% SOLUCIÓN COMPLETA — Actividad Computacional: Proyección Ortogonal
%                      y Sistemas de Recomendación con IA
% Álgebra Lineal — Semestre 2026-1A
% ═══════════════════════════════════════════════════════════════════════════
\documentclass[12pt,a4paper]{article}

% ─── Paquetes ───
\usepackage[utf8]{inputenc}
\usepackage[T1]{fontenc}
\usepackage[spanish]{babel}
\usepackage{amsmath, amssymb, amsthm}
\usepackage{geometry}
\usepackage{graphicx}
\usepackage{xcolor}
\usepackage{tcolorbox}
\usepackage{booktabs}
\usepackage{array}
\usepackage{enumitem}
\usepackage{hyperref}
\usepackage{tikz}
\usetikzlibrary{arrows.meta, calc, decorations.pathreplacing}
\usepackage{fancyhdr}
\usepackage{titlesec}
\usepackage{multicol}

% ─── Geometría ───
\geometry{margin=2.5cm, top=3cm, bottom=3cm}

% ─── Colores personalizados ───
\definecolor{azulvec}{HTML}{2196F3}
\definecolor{rojovec}{HTML}{F44336}
\definecolor{verdeproj}{HTML}{4CAF50}
\definecolor{naranja}{HTML}{FF9800}
\definecolor{fondogris}{HTML}{F5F5F5}
\definecolor{accentblue}{HTML}{1565C0}

% ─── Cajas de tcolorbox ───
\tcbuselibrary{skins, breakable}

\newtcolorbox{solucionbox}[1][]{%
  enhanced, breakable,
  colback=green!3, colframe=verdeproj!80!black,
  fonttitle=\bfseries, title={#1},
  boxrule=1pt, arc=3pt,
  left=6pt, right=6pt, top=6pt, bottom=6pt
}

\newtcolorbox{calcbox}[1][]{%
  enhanced, breakable,
  colback=blue!3, colframe=accentblue!80,
  fonttitle=\bfseries, title={#1},
  boxrule=0.8pt, arc=2pt,
  left=6pt, right=6pt, top=4pt, bottom=4pt
}

\newtcolorbox{notabox}[1][]{%
  enhanced,
  colback=naranja!8, colframe=naranja!80!black,
  fonttitle=\bfseries, title={#1},
  boxrule=0.8pt, arc=2pt,
  left=6pt, right=6pt
}

% ─── Encabezado ───
\pagestyle{fancy}
\fancyhf{}
\fancyhead[L]{\small Álgebra Lineal — 2026-1A}
\fancyhead[R]{\small Solución Actividad Computacional}
\fancyfoot[C]{\thepage}
\renewcommand{\headrulewidth}{0.4pt}

% ─── Comandos ───
\newcommand{\vect}[1]{\vec{#1}}
\newcommand{\proy}{\operatorname{proy}}
\newcommand{\comp}{\operatorname{comp}}

% ═══════════════════════════════════════════════════════════════════════════
\begin{document}

% ─── Portada ───
\begin{titlepage}
\centering
\vspace*{2cm}
{\Huge\bfseries Solución Completa\\[0.3cm]}
{\LARGE Actividad Computacional\\[0.5cm]}
{\Large\color{accentblue} Proyección Ortogonal y Sistemas\\de Recomendación con IA\\[1.5cm]}
\rule{\textwidth}{1pt}\\[0.5cm]
{\large Álgebra Lineal — Semestre 2026-1A\\[0.3cm]}
{\large\today}\\[2cm]

\begin{tcolorbox}[colback=fondogris, colframe=gray!50, width=0.85\textwidth, arc=5pt]
\textbf{Contenido:}
\begin{itemize}[nosep]
  \item Solución detallada de las 4 preguntas
  \item Todos los cálculos explícitos paso a paso
  \item Interpretaciones geométricas y de aplicación
  \item Conclusión completa
\end{itemize}
\end{tcolorbox}
\vfill
\end{titlepage}

\tableofcontents
\newpage

% ═══════════════════════════════════════════════════════════════════════════
\section{Datos de la Actividad}
% ═══════════════════════════════════════════════════════════════════════════

\subsection{Base de datos de películas}

Cada película se representa como un vector en $\mathbb{R}^2$, donde la primera componente mide \textbf{Acción} y la segunda mide \textbf{Romance}.

\begin{center}
\begin{tabular}{lccl}
\toprule
\textbf{Película} & \textbf{Acción} & \textbf{Romance} & \textbf{Vector} \\
\midrule
Terminator           & 5.0 & 1.0 & $\vect{v} = (5,\; 1)$     \\
Titanic              & 1.0 & 5.0 & $\vect{v} = (1,\; 5)$     \\
Sr.\ y Sra.\ Smith   & 4.0 & 4.0 & $\vect{v} = (4,\; 4)$     \\
John Wick            & 5.0 & 0.5 & $\vect{v} = (5,\; 0.5)$   \\
Orgullo y Prejuicio  & 0.5 & 5.0 & $\vect{v} = (0.5,\; 5)$   \\
Matrix               & 4.5 & 1.5 & $\vect{v} = (4.5,\; 1.5)$ \\
Diario de una Pasión & 0.5 & 4.5 & $\vect{v} = (0.5,\; 4.5)$ \\
Misión Imposible     & 4.0 & 2.0 & $\vect{v} = (4,\; 2)$     \\
\bottomrule
\end{tabular}
\end{center}

\subsection{Fórmulas utilizadas}

\begin{notabox}[Recordatorio matemático]
Dados $\vect{a}, \vect{b} \in \mathbb{R}^2$ con $\vect{b} \neq \vect{0}$:

\begin{enumerate}[label=\textbf{\arabic*.}]
\item \textbf{Producto punto:} $\vect{a} \cdot \vect{b} = a_1 b_1 + a_2 b_2$
\item \textbf{Norma:} $\|\vect{a}\| = \sqrt{a_1^2 + a_2^2}$
\item \textbf{Escalar de proyección:} $\dfrac{\vect{a}\cdot\vect{b}}{\|\vect{b}\|^2}$
\item \textbf{Proyección ortogonal:} $\proy_{\vect{b}}(\vect{a}) = \dfrac{\vect{a}\cdot\vect{b}}{\vect{b}\cdot\vect{b}}\;\vect{b}$
\item \textbf{Componente escalar:} $\comp_{\vect{b}}(\vect{a}) = \dfrac{\vect{a}\cdot\vect{b}}{\|\vect{b}\|}$
\item \textbf{Similitud coseno:} $\cos(\theta) = \dfrac{\vect{a}\cdot\vect{b}}{\|\vect{a}\|\;\|\vect{b}\|}$
\end{enumerate}
\end{notabox}

\newpage
% ═══════════════════════════════════════════════════════════════════════════
\section{Pregunta 1 — Interpretación Geométrica de la Proyección}
% ═══════════════════════════════════════════════════════════════════════════

\subsection{Experimento A: Vectores en la misma dirección}

\subsubsection{Caso 1: $\vect{a} = (4,\,1)$, $\vect{b} = (4,\,1)$ (vectores idénticos)}

\begin{calcbox}[Cálculos paso a paso]
\textbf{Paso 1:} Producto punto
\[
\vect{a}\cdot\vect{b} = (4)(4) + (1)(1) = 16 + 1 = 17
\]

\textbf{Paso 2:} Normas
\[
\|\vect{a}\| = \sqrt{4^2 + 1^2} = \sqrt{17} \approx 4.1231
\]
\[
\|\vect{b}\| = \sqrt{4^2 + 1^2} = \sqrt{17} \approx 4.1231
\]
\[
\|\vect{b}\|^2 = 17
\]

\textbf{Paso 3:} Escalar de proyección
\[
\frac{\vect{a}\cdot\vect{b}}{\|\vect{b}\|^2} = \frac{17}{17} = 1
\]

\textbf{Paso 4:} Vector proyección
\[
\proy_{\vect{b}}(\vect{a}) = 1 \times (4,\,1) = (4,\, 1) = \vect{a}
\]
La proyección es \textbf{exactamente igual} al vector $\vect{a}$.

\textbf{Paso 5:} Componente escalar
\[
\comp_{\vect{b}}(\vect{a}) = \frac{17}{\sqrt{17}} = \sqrt{17} \approx 4.1231
\]

\textbf{Paso 6:} Similitud coseno y ángulo
\[
\cos(\theta) = \frac{17}{\sqrt{17}\cdot\sqrt{17}} = \frac{17}{17} = 1 \qquad \Longrightarrow \qquad \theta = 0°
\]

\textbf{Componente perpendicular:}
\[
\vect{a} - \proy_{\vect{b}}(\vect{a}) = (4,1) - (4,1) = (0,\, 0) = \vect{0}
\]
\end{calcbox}

\subsubsection{Caso 2: $\vect{a} = (2,\,0.5)$, $\vect{b} = (4,\,1)$ (misma dirección, diferente magnitud)}

\begin{calcbox}[Cálculos paso a paso]
Observamos que $\vect{a} = (2,\, 0.5) = \frac{1}{2}(4,\, 1) = \frac{1}{2}\vect{b}$.

\textbf{Paso 1:} Producto punto
\[
\vect{a}\cdot\vect{b} = (2)(4) + (0.5)(1) = 8 + 0.5 = 8.5
\]

\textbf{Paso 2:} Normas
\[
\|\vect{a}\| = \sqrt{4 + 0.25} = \sqrt{4.25} \approx 2.0616, \qquad
\|\vect{b}\| = \sqrt{17} \approx 4.1231
\]

\textbf{Paso 3:} Escalar de proyección
\[
\frac{8.5}{17} = 0.5
\]

\textbf{Paso 4:} Vector proyección
\[
\proy_{\vect{b}}(\vect{a}) = 0.5 \times (4,\, 1) = (2,\, 0.5) = \vect{a}
\]

\textbf{Paso 5 y 6:}
\[
\cos(\theta) = \frac{8.5}{2.0616 \times 4.1231} = \frac{8.5}{8.5} = 1, \qquad \theta = 0°
\]

\textbf{Componente perpendicular:} $(2, 0.5) - (2, 0.5) = (0, 0) = \vect{0}$
\end{calcbox}

% ── Gráfica TikZ Experimento A ──
\begin{center}
\begin{tikzpicture}[scale=1.1]
  \draw[gray!30, thin] (-0.5,-0.5) grid (5.5,2.5);
  \draw[->, thick] (-0.5,0) -- (5.5,0) node[right]{Acción};
  \draw[->, thick] (0,-0.5) -- (0,2.5) node[above]{Romance};
  % Vector b
  \draw[-{Stealth[length=8pt]}, rojovec, line width=2pt] (0,0) -- (4,1)
    node[above right, font=\small\bfseries]{$\vect{b}=(4,1)$};
  % Vector a (caso 1 — coincide con b)
  \draw[-{Stealth[length=8pt]}, azulvec, line width=2pt, dashed] (0,0) -- (4,1);
  % Vector a caso 2
  \draw[-{Stealth[length=8pt]}, azulvec!70, line width=2pt] (0,0) -- (2,0.5)
    node[below right, font=\small]{$\vect{a}=(2,0.5)$};
  % Proyección caso 2 (coincide con a)
  \draw[-{Stealth[length=6pt]}, verdeproj, line width=1.5pt] (0,0) -- (2,0.5);
  \node[font=\footnotesize, text=verdeproj] at (1.3, -0.3) {proy $= \vect{a}$};
\end{tikzpicture}
\end{center}

\subsection{Experimento B: Vectores perpendiculares}

$\vect{a} = (-1,\,4)$, $\vect{b} = (4,\,1)$.

\begin{calcbox}[Cálculos paso a paso]
\textbf{Paso 1:} Producto punto
\[
\vect{a}\cdot\vect{b} = (-1)(4) + (4)(1) = -4 + 4 = 0
\]
El producto punto es \textbf{cero}; los vectores son perpendiculares.

\textbf{Paso 2:} Normas
\[
\|\vect{a}\| = \sqrt{1+16} = \sqrt{17} \approx 4.1231, \qquad
\|\vect{b}\| = \sqrt{17} \approx 4.1231
\]

\textbf{Paso 3:} Escalar de proyección
\[
\frac{0}{17} = 0
\]

\textbf{Paso 4:} Vector proyección
\[
\proy_{\vect{b}}(\vect{a}) = 0 \times (4,\,1) = \vect{0} = (0,\,0)
\]
La proyección es el \textbf{vector cero}.

\textbf{Paso 5:} Componente escalar
\[
\comp_{\vect{b}}(\vect{a}) = \frac{0}{\sqrt{17}} = 0
\]

\textbf{Paso 6:} Similitud coseno y ángulo
\[
\cos(\theta) = \frac{0}{\sqrt{17}\cdot\sqrt{17}} = 0 \qquad \Longrightarrow \qquad \theta = 90°
\]

\textbf{Componente perpendicular:}
\[
\vect{a} - \vect{0} = (-1,\, 4) 
\]
El componente perpendicular es \textbf{todo} el vector $\vect{a}$.
\end{calcbox}

\begin{center}
\begin{tikzpicture}[scale=0.9]
  \draw[gray!30, thin] (-2.5,-0.5) grid (5.5,5.5);
  \draw[->, thick] (-2.5,0) -- (5.5,0) node[right]{Acción};
  \draw[->, thick] (0,-0.5) -- (0,5.5) node[above]{Romance};
  % Vector b
  \draw[-{Stealth[length=8pt]}, rojovec, line width=2pt] (0,0) -- (4,1)
    node[below right, font=\small\bfseries]{$\vect{b}=(4,1)$};
  % Vector a
  \draw[-{Stealth[length=8pt]}, azulvec, line width=2pt] (0,0) -- (-1,4)
    node[above left, font=\small\bfseries]{$\vect{a}=(-1,4)$};
  % Proyección = origen (punto verde)
  \fill[verdeproj] (0,0) circle (4pt);
  \node[font=\footnotesize, verdeproj, below left] at (-0.1,-0.1) {proy $= \vect{0}$};
  % Ángulo recto
  \draw[thick] (0.6,0.15) -- (0.45,0.75) -- (-0.15, 0.6);
  \node[font=\footnotesize] at (1.2, 0.7) {$90°$};
\end{tikzpicture}
\end{center}

\subsection{Experimento C: Vectores en direcciones distintas}

$\vect{a} = (3,\,4)$, $\vect{b} = (4,\,1)$.

\begin{calcbox}[Cálculos paso a paso]
\textbf{Paso 1:} Producto punto
\[
\vect{a}\cdot\vect{b} = (3)(4) + (4)(1) = 12 + 4 = 16
\]

\textbf{Paso 2:} Normas
\[
\|\vect{a}\| = \sqrt{9 + 16} = \sqrt{25} = 5, \qquad
\|\vect{b}\| = \sqrt{17} \approx 4.1231, \qquad
\|\vect{b}\|^2 = 17
\]

\textbf{Paso 3:} Escalar de proyección
\[
\frac{16}{17} \approx 0.9412
\]

\textbf{Paso 4:} Vector proyección
\[
\proy_{\vect{b}}(\vect{a}) = \frac{16}{17} \times (4,\, 1) = \left(\frac{64}{17},\; \frac{16}{17}\right) \approx (3.7647,\; 0.9412)
\]

\textbf{Paso 5:} Componente escalar
\[
\comp_{\vect{b}}(\vect{a}) = \frac{16}{\sqrt{17}} \approx 3.8806
\]

\textbf{Paso 6:} Similitud coseno y ángulo
\[
\cos(\theta) = \frac{16}{5 \times \sqrt{17}} = \frac{16}{20.6155} \approx 0.7761
\]
\[
\theta = \arccos(0.7761) \approx 39.1°
\]

\textbf{Componente perpendicular:}
\[
\vect{a} - \proy_{\vect{b}}(\vect{a}) = (3, 4) - (3.7647,\, 0.9412) = (-0.7647,\; 3.0588)
\]
\end{calcbox}

\begin{center}
\begin{tikzpicture}[scale=1.0]
  \draw[gray!30, thin] (-1,-0.5) grid (5.5,5.5);
  \draw[->, thick] (-1,0) -- (5.5,0) node[right]{Acción};
  \draw[->, thick] (0,-0.5) -- (0,5.5) node[above]{Romance};
  % Vector b
  \draw[-{Stealth[length=8pt]}, rojovec, line width=2pt] (0,0) -- (4,1)
    node[below right, font=\small\bfseries]{$\vect{b}=(4,1)$};
  % Vector a
  \draw[-{Stealth[length=8pt]}, azulvec, line width=2pt] (0,0) -- (3,4)
    node[above right, font=\small\bfseries]{$\vect{a}=(3,4)$};
  % Proyección
  \draw[-{Stealth[length=7pt]}, verdeproj, line width=2pt] (0,0) -- (3.7647,0.9412)
    node[below, font=\small, yshift=-5pt]{proy $\approx (3.76, 0.94)$};
  % Componente perpendicular (línea punteada)
  \draw[naranja, line width=1.5pt, dashed] (3.7647,0.9412) -- (3,4)
    node[midway, right, font=\footnotesize]{perp};
\end{tikzpicture}
\end{center}

\subsection{Respuestas a la Pregunta 1}

\begin{solucionbox}[Respuesta 1a — Misma dirección]
Cuando $\vect{a}$ y $\vect{b}$ apuntan en la \textbf{misma dirección}:
\begin{itemize}
  \item El vector de proyección (verde) \textbf{coincide exactamente} con $\vect{a}$. Toda la ``información'' de $\vect{a}$ va en la dirección de $\vect{b}$.
  \item El componente perpendicular (naranja) es el \textbf{vector cero} $(0, 0)$: no hay componente que ``se desvíe'' de la dirección de $\vect{b}$.
  \item El ángulo entre los vectores es $\theta = 0°$ y la similitud coseno es $\cos(0°) = 1$ (máxima similitud posible).
\end{itemize}
Esto ocurre independientemente de la magnitud de $\vect{a}$: ya sea $\vect{a} = (4,1)$ o $\vect{a} = (2, 0.5)$, al estar en la misma dirección que $\vect{b}$, la proyección siempre reproduce a $\vect{a}$.
\end{solucionbox}

\begin{solucionbox}[Respuesta 1b — Perpendiculares]
Cuando $\vect{a}$ y $\vect{b}$ son \textbf{perpendiculares} (ortogonales):
\begin{itemize}
  \item El vector de proyección (verde) \textbf{desaparece}: se colapsa al vector cero $\vect{0} = (0,0)$.
  \item El producto punto vale exactamente $\vect{a}\cdot\vect{b} = (-1)(4) + (4)(1) = 0$.
  \item La similitud coseno vale $\cos(90°) = 0$.
\end{itemize}
Geométricamente, $\vect{a}$ no tiene \emph{ninguna} componente en la dirección de $\vect{b}$; toda su magnitud es perpendicular. El componente perpendicular (naranja) es todo $\vect{a}$.
\end{solucionbox}

\begin{solucionbox}[Respuesta 1c — Proyección cercana a cero entre películas]
Si la proyección de la película del usuario sobre una película candidata es \textbf{cercana a cero}, significa que las dos películas \textbf{no comparten características relevantes}: apuntan en direcciones muy diferentes en el espacio de géneros. La candidata no tiene casi nada en común con lo que le gusta al usuario.

\textbf{No se recomendaría} esa película, porque una proyección cercana a cero implica similitud coseno cercana a cero (ángulo $\approx 90°$). Para un fan de acción pura, una comedia romántica pura sería casi ``invisible'' en términos de proyección: la componente de la película del usuario que va en la dirección de la candidata es despreciable.
\end{solucionbox}

\newpage
% ═══════════════════════════════════════════════════════════════════════════
\section{Pregunta 2 — Proyección y Similitud entre Películas}
% ═══════════════════════════════════════════════════════════════════════════

\subsection{Cálculos detallados para cada par}

% ── Par 1 ──
\subsubsection{Par 1: Terminator $(5,\,1)$ vs John Wick $(5,\,0.5)$}

\begin{calcbox}[Cálculos]
\[
\vect{a}\cdot\vect{b} = (5)(5) + (1)(0.5) = 25 + 0.5 = 25.5
\]
\[
\|\vect{a}\| = \sqrt{25+1} = \sqrt{26} \approx 5.0990, \qquad
\|\vect{b}\| = \sqrt{25+0.25} = \sqrt{25.25} \approx 5.0249
\]
\[
\comp_{\vect{b}}(\vect{a}) = \frac{25.5}{\sqrt{25.25}} = \frac{25.5}{5.0249} \approx 5.0746
\]
\[
\cos(\theta) = \frac{25.5}{\sqrt{26}\cdot\sqrt{25.25}} = \frac{25.5}{25.6244} \approx 0.9951
\]
\[
\theta = \arccos(0.9951) \approx 5.7°
\]
\end{calcbox}

% ── Par 2 ──
\subsubsection{Par 2: Terminator $(5,\,1)$ vs Titanic $(1,\,5)$}

\begin{calcbox}[Cálculos]
\[
\vect{a}\cdot\vect{b} = (5)(1) + (1)(5) = 5 + 5 = 10
\]
\[
\|\vect{a}\| = \sqrt{26} \approx 5.0990, \qquad
\|\vect{b}\| = \sqrt{1+25} = \sqrt{26} \approx 5.0990
\]
\[
\comp_{\vect{b}}(\vect{a}) = \frac{10}{\sqrt{26}} \approx 1.9612
\]
\[
\cos(\theta) = \frac{10}{26} \approx 0.3846
\]
\[
\theta = \arccos(0.3846) \approx 67.4°
\]
\end{calcbox}

% ── Par 3 ──
\subsubsection{Par 3: Sr.\ y Sra.\ Smith $(4,\,4)$ vs Matrix $(4.5,\,1.5)$}

\begin{calcbox}[Cálculos]
\[
\vect{a}\cdot\vect{b} = (4)(4.5) + (4)(1.5) = 18 + 6 = 24
\]
\[
\|\vect{a}\| = \sqrt{16+16} = \sqrt{32} = 4\sqrt{2} \approx 5.6569, \qquad
\|\vect{b}\| = \sqrt{20.25+2.25} = \sqrt{22.5} \approx 4.7434
\]
\[
\comp_{\vect{b}}(\vect{a}) = \frac{24}{\sqrt{22.5}} \approx 5.0596
\]
\[
\cos(\theta) = \frac{24}{\sqrt{32}\cdot\sqrt{22.5}} = \frac{24}{\sqrt{720}} = \frac{24}{26.8328} \approx 0.8944
\]
\[
\theta = \arccos(0.8944) \approx 26.6°
\]
\end{calcbox}

% ── Par 4 ──
\subsubsection{Par 4: Orgullo y Prejuicio $(0.5,\,5)$ vs Diario de una Pasión $(0.5,\,4.5)$}

\begin{calcbox}[Cálculos]
\[
\vect{a}\cdot\vect{b} = (0.5)(0.5) + (5)(4.5) = 0.25 + 22.5 = 22.75
\]
\[
\|\vect{a}\| = \sqrt{0.25+25} = \sqrt{25.25} \approx 5.0249, \qquad
\|\vect{b}\| = \sqrt{0.25+20.25} = \sqrt{20.5} \approx 4.5277
\]
\[
\comp_{\vect{b}}(\vect{a}) = \frac{22.75}{\sqrt{20.5}} \approx 5.0248
\]
\[
\cos(\theta) = \frac{22.75}{\sqrt{25.25}\cdot\sqrt{20.5}} = \frac{22.75}{\sqrt{517.625}} = \frac{22.75}{22.7514} \approx 0.9999
\]
\[
\theta = \arccos(0.9999) \approx 0.8°
\]
\end{calcbox}

% ── Par 5 ──
\subsubsection{Par 5: Terminator $(5,\,1)$ vs Sr.\ y Sra.\ Smith $(4,\,4)$}

\begin{calcbox}[Cálculos]
\[
\vect{a}\cdot\vect{b} = (5)(4) + (1)(4) = 20 + 4 = 24
\]
\[
\|\vect{a}\| = \sqrt{26} \approx 5.0990, \qquad
\|\vect{b}\| = \sqrt{32} \approx 5.6569
\]
\[
\comp_{\vect{b}}(\vect{a}) = \frac{24}{\sqrt{32}} = \frac{24}{4\sqrt{2}} = 3\sqrt{2} \approx 4.2426
\]
\[
\cos(\theta) = \frac{24}{\sqrt{26}\cdot\sqrt{32}} = \frac{24}{\sqrt{832}} = \frac{24}{28.8444} \approx 0.8321
\]
\[
\theta = \arccos(0.8321) \approx 33.7°
\]
\end{calcbox}

\subsection{Tabla completada}

\begin{center}
\renewcommand{\arraystretch}{1.4}
\begin{tabular}{c l c c c}
\toprule
\textbf{\#} & \textbf{Par de películas} & \textbf{Comp.\ escalar} & \textbf{Similitud coseno} & \textbf{Ángulo} \\
\midrule
1 & Terminator vs John Wick                    & $5.0746$  & $0.9951$ & $5.7°$  \\
2 & Terminator vs Titanic                      & $1.9612$  & $0.3846$ & $67.4°$ \\
3 & Sr.\ y Sra.\ Smith vs Matrix               & $5.0596$  & $0.8944$ & $26.6°$ \\
4 & Orgullo y Prejuicio vs Diario de una Pasión & $5.0248$  & $0.9999$ & $0.8°$  \\
5 & Terminator vs Sr.\ y Sra.\ Smith            & $4.2426$  & $0.8321$ & $33.7°$ \\
\bottomrule
\end{tabular}
\end{center}

\subsection{Respuestas a la Pregunta 2}

\begin{solucionbox}[Respuesta 2a — Mayor similitud coseno]
El par con la \textbf{mayor similitud coseno} es \textbf{Orgullo y Prejuicio vs Diario de una Pasión}, con $\cos(\theta) \approx 0.9999$ y un ángulo de apenas $0.8°$.

Esto es \textbf{completamente coherente} con la intuición: ambas son películas altamente románticas con muy poca acción. Sus vectores $(0.5,\, 5)$ y $(0.5,\, 4.5)$ apuntan casi exactamente en la misma dirección (prácticamente verticales, hacia el eje de Romance). Son las películas más parecidas de todo el catálogo.

El segundo par más similar es Terminator vs John Wick ($\cos \approx 0.9951$), lo cual también tiene sentido: ambas son películas de acción intensa con mínimo romance.
\end{solucionbox}

\begin{solucionbox}[Respuesta 2b — Relación entre componente escalar y similitud coseno]
El componente escalar y la similitud coseno se \textbf{comportan de manera similar}: cuando uno crece, el otro tiende a crecer. Sin embargo, \textbf{no son iguales} y pueden diferir significativamente en valor numérico.

La razón es que están relacionados pero miden cosas distintas:
\[
\comp_{\vect{b}}(\vect{a}) = \frac{\vect{a}\cdot\vect{b}}{\|\vect{b}\|}, \qquad\qquad
\cos(\theta) = \frac{\vect{a}\cdot\vect{b}}{\|\vect{a}\|\cdot\|\vect{b}\|}
\]

La diferencia es el factor $\|\vect{a}\|$ en el denominador de la similitud coseno:
\[
\cos(\theta) = \frac{\comp_{\vect{b}}(\vect{a})}{\|\vect{a}\|}
\]

El componente escalar depende de la \textbf{magnitud} de $\vect{a}$ (películas con puntajes más altos producen componentes mayores), mientras que la similitud coseno \textbf{normaliza} por ambas magnitudes, midiendo solo la \emph{dirección} relativa. Por ejemplo, en el Par 2, el componente escalar es $1.96$ pero $\cos(\theta) = 0.38$; la similitud coseno revela que a pesar de ser largo el vector, las direcciones son bastante distintas.
\end{solucionbox}

\begin{solucionbox}[Respuesta 2c — Explicación en palabras simples]
Imagina que cada película es una \textbf{flecha en un mapa} que indica ``hacia dónde va'' en términos de acción y romance. La proyección ortogonal es como preguntar: \emph{``¿cuánto de lo que le gustó al usuario apunta en la misma dirección que esta otra película?''}

Es como si tuvieras una \textbf{linterna} apuntando en la dirección de la película candidata: la ``sombra'' que la flecha del usuario proyecta sobre esa dirección te dice cuánto se parecen. Si la sombra es larga, las películas van por el mismo camino. Si la sombra es corta o desaparece, las películas van por caminos muy diferentes, como si la flecha del usuario pasara ``de largo'' sin tocar la dirección de la candidata.
\end{solucionbox}

\newpage
% ═══════════════════════════════════════════════════════════════════════════
\section{Pregunta 3 — Ortogonalidad y su Significado en Machine Learning}
% ═══════════════════════════════════════════════════════════════════════════

\subsection{Análisis con el explorador de ángulos}

En esta parte, el vector $\vect{b} = (5, 1)$ (Terminator) está fijo, y la ``Película X'' se rota un ángulo $\alpha$ respecto a Terminator, con magnitud $\|\text{Película X}\| = 4$.

\begin{center}
\renewcommand{\arraystretch}{1.3}
\begin{tabular}{c c c c c}
\toprule
\textbf{Ángulo $\alpha$} & \textbf{$\vect{a}\cdot\vect{b}$} & $\cos(\theta)$ & \textbf{Proyección} & \textbf{Observación} \\
\midrule
$0°$   & $20.0$ & $1.000$   & Máxima       & Misma dirección \\
$45°$  & $14.14$ & $0.707$  & Moderada-alta & Dirección intermedia \\
$90°$  & $0$    & $0.000$   & $\vect{0}$   & \textbf{Perpendiculares} \\
$135°$ & $-14.14$ & $-0.707$ & Negativa   & Más opuestos que similares \\
$180°$ & $-20.0$ & $-1.000$ & Mínima       & Direcciones opuestas \\
\bottomrule
\end{tabular}
\end{center}

\begin{center}
\begin{tikzpicture}[scale=0.65]
  % Ejes
  \draw[gray!20, thin] (-5.5,-5.5) grid (6.5,5.5);
  \draw[->, thick] (-5.5,0) -- (6.5,0) node[right]{Acción};
  \draw[->, thick] (0,-5.5) -- (0,5.5) node[above]{Romance};
  
  % Terminator (fijo)
  \draw[-{Stealth[length=7pt]}, rojovec, line width=2.5pt] (0,0) -- (5,1)
    node[right, font=\small\bfseries]{Terminator $(5,1)$};
  
  % Ángulo de Terminator: arctan(1/5) ≈ 11.31°
  % 0° → misma dir (5,1) normalizado × 4 ≈ (3.92, 0.78)
  \draw[-{Stealth[length=6pt]}, azulvec!80, line width=1.5pt] (0,0) -- (3.92,0.78)
    node[below right, font=\tiny]{$0°$};
  % 45° → ángulo total ≈ 56.3° → (4cos56.3, 4sin56.3) ≈ (2.22, 3.33)
  \draw[-{Stealth[length=6pt]}, azulvec!60, line width=1.5pt] (0,0) -- (2.22,3.33)
    node[above right, font=\tiny]{$45°$};
  % 90° → ángulo total ≈ 101.3° → (4cos101.3, 4sin101.3) ≈ (-0.78, 3.92)
  \draw[-{Stealth[length=6pt]}, verdeproj!80, line width=1.5pt] (0,0) -- (-0.78,3.92)
    node[above left, font=\tiny]{$90°$ (ortogonal)};
  % 135° → ángulo total ≈ 146.3°
  \draw[-{Stealth[length=6pt]}, naranja!80, line width=1.5pt] (0,0) -- (-3.33,2.22)
    node[left, font=\tiny]{$135°$};
  % 180° → ángulo total ≈ 191.3° → dirección opuesta
  \draw[-{Stealth[length=6pt]}, rojovec!60, line width=1.5pt] (0,0) -- (-3.92,-0.78)
    node[below left, font=\tiny]{$180°$};
  
  % Arco de ángulo
  \draw[thick, ->] (2,0.4) arc[start angle=11.3, end angle=101.3, radius=2.04];
  \node[font=\footnotesize] at (1.8, 2.2) {$\alpha$};
\end{tikzpicture}
\end{center}

\subsection{Respuestas a la Pregunta 3}

\begin{solucionbox}[Respuesta 3a — Ángulo donde la proyección desaparece]
El vector de proyección (verde) \textbf{desaparece} (se vuelve el vector cero) exactamente cuando el ángulo es $\theta = 90°$.

En ese punto:
\begin{itemize}
  \item Producto punto: $\vect{a} \cdot \vect{b} = 0$
  \item Similitud coseno: $\cos(90°) = 0$
  \item Componente escalar de la proyección: $0$
\end{itemize}

Este es el punto de \textbf{ortogonalidad}: los vectores son perpendiculares y no comparten \emph{ninguna} componente en la dirección del otro.
\end{solucionbox}

\begin{solucionbox}[Respuesta 3b — Más allá de 90°]
Cuando el ángulo \textbf{supera los $90°$}, el componente escalar de la proyección se vuelve \textbf{negativo}. La similitud coseno también es negativa.

En el contexto de películas, una ``similitud negativa'' significaría que la candidata tiene características \textbf{opuestas} a lo que busca el usuario. Sin embargo, en la práctica tiene \textbf{sentido limitado}: los puntajes de películas son positivos (Acción $\geq 0$, Romance $\geq 0$), así que los vectores están en el primer cuadrante y no se alcanzan ángulos mayores a $90°$. En sistemas más complejos con vectores que sí pueden ser negativos (como en análisis de sentimiento), la similitud negativa indica rechazo u oposición.
\end{solucionbox}

\begin{solucionbox}[Respuesta 3c — Película perpendicular a Terminator]
Un vector perpendicular a Terminator $(5, 1)$ es $(-1, 5)$ (verificación: $(5)(-1) + (1)(5) = 0$). Sin embargo, como los puntajes no pueden ser negativos, la película más cercana a esta dirección perpendicular en el primer cuadrante sería una con \textbf{mucho romance y muy poca acción}, como $\approx (0.5,\, 5)$.

De hecho, \textbf{Orgullo y Prejuicio} $(0.5,\, 5)$ es casi perpendicular a Terminator:
\[
(5)(0.5) + (1)(5) = 2.5 + 5 = 7.5 \qquad (\text{producto punto pequeño, no exactamente cero})
\]
\[
\cos(\theta) = \frac{7.5}{\sqrt{26}\cdot\sqrt{25.25}} \approx \frac{7.5}{25.62} \approx 0.2927 \qquad \theta \approx 73°
\]

\textbf{No tendría sentido} recomendarla a un fan de Terminator. La proyección de Terminator sobre Orgullo y Prejuicio es muy pequeña, lo que significa que casi nada de lo que define a Terminator (acción intensa) se ``alinea'' con lo que define a Orgullo y Prejuicio (romance puro). Con la proyección tan cercana a cero, el sistema de recomendación la colocaría en los últimos lugares del ranking.
\end{solucionbox}

\newpage
% ═══════════════════════════════════════════════════════════════════════════
\section{Pregunta 4 — Análisis del Sistema de Recomendación}
% ═══════════════════════════════════════════════════════════════════════════

\subsection{Respuesta 4a — Ranking para fan de Terminator}

\begin{calcbox}[Cálculos: Similitud coseno de Terminator $(5, 1)$ con todas las demás]
Para cada película candidata $\vect{c}$, calculamos $\cos(\theta) = \dfrac{(5,1)\cdot \vect{c}}{\sqrt{26}\;\|\vect{c}\|}$:

\medskip
\renewcommand{\arraystretch}{1.3}
\begin{tabular}{l c c c c}
\toprule
\textbf{Candidata} & \textbf{Vector} & $\vect{a}\cdot\vect{c}$ & $\cos(\theta)$ & $\theta$ \\
\midrule
John Wick            & $(5, 0.5)$   & $25.5$  & $0.9951$ & $5.7°$  \\
Matrix               & $(4.5, 1.5)$ & $24.0$  & $0.9923$ & $7.1°$  \\
Misión Imposible     & $(4, 2)$     & $22.0$  & $0.9648$ & $15.3°$ \\
Sr.\ y Sra.\ Smith   & $(4, 4)$     & $24.0$  & $0.8321$ & $33.7°$ \\
Titanic              & $(1, 5)$     & $10.0$  & $0.3846$ & $67.4°$ \\
Diario de una Pasión & $(0.5, 4.5)$ & $7.0$   & $0.3032$ & $72.4°$ \\
Orgullo y Prejuicio  & $(0.5, 5)$   & $7.5$   & $0.2927$ & $73.0°$ \\
\bottomrule
\end{tabular}
\end{calcbox}

\begin{solucionbox}[Respuesta 4a]
Las \textbf{3 películas más recomendadas} para un fan de Terminator son:

\begin{enumerate}
  \item[\textbf{🥇}] \textbf{John Wick} $(\cos = 0.9951)$
  \item[\textbf{🥈}] \textbf{Matrix} $(\cos = 0.9923)$
  \item[\textbf{🥉}] \textbf{Misión Imposible} $(\cos = 0.9648)$
\end{enumerate}

\textbf{Sí tiene sentido.} Las tres son películas de \textbf{acción intensa con poco romance}, al igual que Terminator. Sus vectores apuntan en una dirección muy cercana a $(5, 1)$: todos están casi sobre el eje horizontal (Acción). El ángulo entre estos vectores y Terminator es menor a $16°$, lo que matemáticamente confirma que sus direcciones son casi idénticas. Un fan de escenas de acción con explosiones y peleas encontrará satisfactorias estas películas.
\end{solucionbox}

\subsection{Respuesta 4b — Ranking para fan de Sr.\ y Sra.\ Smith}

\begin{calcbox}[Cálculos: Similitud coseno de Sr.\ y Sra.\ Smith $(4, 4)$ con todas]
\renewcommand{\arraystretch}{1.3}
\begin{tabular}{l c c c}
\toprule
\textbf{Candidata} & \textbf{Vector} & $\cos(\theta)$ & $\theta$ \\
\midrule
Misión Imposible     & $(4, 2)$     & $0.9487$ & $18.4°$ \\
Matrix               & $(4.5, 1.5)$ & $0.8944$ & $26.6°$ \\
Terminator           & $(5, 1)$     & $0.8321$ & $33.7°$ \\
Titanic              & $(1, 5)$     & $0.8321$ & $33.7°$ \\
Diario de una Pasión & $(0.5, 4.5)$ & $0.7810$ & $38.7°$ \\
John Wick            & $(5, 0.5)$   & $0.7740$ & $39.2°$ \\
Orgullo y Prejuicio  & $(0.5, 5)$   & $0.7740$ & $39.2°$ \\
\bottomrule
\end{tabular}
\end{calcbox}

\begin{solucionbox}[Respuesta 4b]
El ranking de Sr.\ y Sra.\ Smith produce recomendaciones \textbf{mucho más variadas} que el de Terminator. Las similitudes van de $0.95$ hasta $0.77$, un rango mucho más estrecho que el de Terminator ($0.99$ a $0.29$).

La razón es \textbf{geométrica}: el vector de Sr.\ y Sra.\ Smith es $(4, 4)$, que apunta exactamente a $45°$ de ambos ejes (Acción y Romance). Al estar en la \emph{bisectriz} del primer cuadrante, este vector mantiene un ángulo moderado con \textbf{todas} las demás películas, sin importar si son de acción pura o romance puro.

Nótese que $\cos(\theta)$ con Terminator $(5,1)$ y con Titanic $(1,5)$ es \textbf{idéntico}: $0.8321$, porque $(4,4)$ está exactamente a la misma distancia angular de ambos. Esto confirma que Sr.\ y Sra.\ Smith es una película ``equilibrada'' que comparte rasgos tanto con las de acción como con las de romance, lo que produce recomendaciones diversas.

En contraste, Terminator $(5,1)$ está muy inclinado hacia la acción, así que es muy similar a otras películas de acción ($\cos \approx 1$) pero muy diferente de las románticas ($\cos \approx 0.3$).
\end{solucionbox}

\subsection{Respuesta 4c — Película menos similar a Titanic}

\begin{calcbox}[Cálculos: Ranking para Titanic $(1, 5)$]
\renewcommand{\arraystretch}{1.3}
\begin{tabular}{l c c c}
\toprule
\textbf{Candidata} & \textbf{Vector} & $\cos(\theta)$ & $\theta$ \\
\midrule
Diario de una Pasión & $(0.5, 4.5)$ & $0.9962$ & $5.0°$  \\
Orgullo y Prejuicio  & $(0.5, 5)$   & $0.9951$ & $5.7°$  \\
Sr.\ y Sra.\ Smith   & $(4, 4)$     & $0.8321$ & $33.7°$ \\
Misión Imposible     & $(4, 2)$     & $0.6140$ & $52.1°$ \\
Matrix               & $(4.5, 1.5)$ & $0.4961$ & $60.3°$ \\
Terminator           & $(5, 1)$     & $0.3846$ & $67.4°$ \\
John Wick            & $(5, 0.5)$   & $0.2927$ & $73.0°$ \\
\bottomrule
\end{tabular}
\end{calcbox}

\begin{solucionbox}[Respuesta 4c]
La película con la \textbf{menor similitud coseno} respecto a Titanic es \textbf{John Wick}, con $\cos(\theta) = 0.2927$ y un ángulo de $73.0°$.

Este valor es \textbf{positivo pero cercano a cero} (no negativo, ya que todos los vectores están en el primer cuadrante con componentes positivas). El ángulo de $73°$ está acercándose a los $90°$ de la ortogonalidad total que exploramos en la Parte 3.

Esto se relaciona directamente con la ortogonalidad: John Wick $(5, 0.5)$ es casi pura acción con mínimo romance, mientras que Titanic $(1, 5)$ es casi puro romance con mínima acción. Sus vectores apuntan en direcciones cuasi-perpendiculares. Si existiera una película con vector exactamente perpendicular a Titanic, tendría similitud coseno $= 0$ y ángulo $= 90°$. John Wick es lo más cercano a esa situación en el catálogo.
\end{solucionbox}

\subsection{Respuesta 4d — Nueva película igualmente similar a Terminator y Titanic}

\begin{solucionbox}[Respuesta 4d]
Para que una película sea \textbf{igualmente similar} a Terminator $(5, 1)$ y a Titanic $(1, 5)$, su vector debe estar en la \textbf{bisectriz} del ángulo que forman ambos. Como Terminator y Titanic son simétricos respecto a la recta $y = x$, la bisectriz es precisamente la dirección $(1, 1)$.

\textbf{Vector propuesto:} $(3,\, 3)$ (o cualquier múltiplo positivo de $(1,1)$, como $(4,4)$).

\medskip
\textbf{Verificación con $\vect{a} = (3, 3)$:}

\begin{itemize}
  \item \textbf{Con Terminator} $(5, 1)$:
  \[
  \cos(\theta) = \frac{(3)(5) + (3)(1)}{\sqrt{9+9}\cdot\sqrt{25+1}} = \frac{15+3}{\sqrt{18}\cdot\sqrt{26}} = \frac{18}{4.2426 \times 5.0990} = \frac{18}{21.6333} \approx 0.8321
  \]

  \item \textbf{Con Titanic} $(1, 5)$:
  \[
  \cos(\theta) = \frac{(3)(1) + (3)(5)}{\sqrt{18}\cdot\sqrt{26}} = \frac{3+15}{\sqrt{18}\cdot\sqrt{26}} = \frac{18}{21.6333} \approx 0.8321
  \]
\end{itemize}

¡Ambas similitudes son \textbf{exactamente iguales}: $\cos(\theta) = 0.8321$!

De hecho, la película \textbf{Sr.\ y Sra.\ Smith} $(4, 4)$ cumple exactamente esta propiedad, pues $(4, 4) = 4 \cdot (1, 1)$ y también está en la dirección de la bisectriz. Esto confirma que es una película equilibrada entre acción y romance, igualmente afín a los fans de cualquiera de los dos géneros.

La demostración algebraica es elegante: si $\vect{a} = (t, t)$ para cualquier $t > 0$:
\[
\cos\bigl(\vect{a},\, (5,1)\bigr) = \frac{5t + t}{\sqrt{2t^2}\cdot\sqrt{26}} = \frac{6t}{t\sqrt{2}\cdot\sqrt{26}} = \frac{6}{\sqrt{52}} = \frac{6}{2\sqrt{13}} = \frac{3}{\sqrt{13}}
\]
\[
\cos\bigl(\vect{a},\, (1,5)\bigr) = \frac{t + 5t}{\sqrt{2t^2}\cdot\sqrt{26}} = \frac{6t}{t\sqrt{2}\cdot\sqrt{26}} = \frac{6}{\sqrt{52}} = \frac{3}{\sqrt{13}}
\]
Ambos dan el mismo resultado $\dfrac{3}{\sqrt{13}} \approx 0.8321$, independientemente de $t$.
\end{solucionbox}

\newpage
% ═══════════════════════════════════════════════════════════════════════════
\section{Conclusión}
% ═══════════════════════════════════════════════════════════════════════════

\begin{solucionbox}[Conclusión de la actividad]
\begin{enumerate}[label=\textbf{\arabic*.}, leftmargin=*, itemsep=8pt]

\item La \textbf{proyección ortogonal} en el contexto de similitud entre películas mide \emph{cuánto del perfil de una película ``va en la dirección'' de otra}. Matemáticamente, es la parte del vector de la película del usuario que se alinea con el vector de la película candidata. Un valor grande significa que ambas comparten las mismas proporciones de acción y romance; un valor cercano a cero significa que las películas tienen perfiles independientes.

\item La proyección indica \textbf{similitud} cuando es grande y positiva (vectores casi paralelos, ángulo $\approx 0°$), indica \textbf{diferencia} cuando es cercana a cero (vectores casi perpendiculares, ángulo $\approx 90°$), e indica \textbf{oposición} cuando es negativa (vectores en sentidos opuestos, ángulo $> 90°$). En nuestro contexto de películas con puntajes no negativos, la similitud siempre es positiva, pero en sistemas reales con componentes arbitrarias sí puede ser negativa.

\item La \textbf{ortogonalidad} es un concepto crucial en machine learning porque identifica cuándo dos elementos \emph{no comparten información alguna}. Si dos vectores son perpendiculares, la proyección de uno sobre otro es cero, lo que significa que conocer uno no te dice nada sobre el otro. Esto es la base de conceptos avanzados como la reducción de dimensionalidad (PCA), la independencia de características, y los espacios de embeddings usados en redes neuronales modernas.

\item La proyección y la similitud coseno están \textbf{íntimamente relacionadas}. La similitud coseno es simplemente el componente escalar de la proyección dividido entre la norma del vector que se proyecta: $\cos(\theta) = \comp_{\vect{b}}(\vect{a})/\|\vect{a}\|$. La ventaja de la similitud coseno es que \emph{normaliza} por las magnitudes de ambos vectores, midiendo únicamente la \emph{dirección} relativa. Esto es deseable en sistemas de recomendación porque dos películas pueden tener puntajes absolutos muy diferentes pero proporciones similares de cada género.

\item Este modelo de solo \textbf{2 dimensiones} es una simplificación pedagógica. Los sistemas reales de IA (Netflix, Spotify, YouTube) representan películas o canciones con vectores de \textbf{cientos o miles de dimensiones}, donde cada componente captura un aspecto sutil del contenido (subgéneros, tono emocional, ritmo narrativo, director, año, etc.). En dimensiones tan altas ya no es posible visualizar los vectores en un plano, pero la matemática subyacente —producto punto, proyección, similitud coseno— es \textbf{exactamente la misma}. Las limitaciones principales del modelo 2D son: (a) solo captura dos géneros de los muchos que existen, (b) no modela interacciones complejas entre categorías, y (c) los puntajes son asignados manualmente en lugar de ser aprendidos a partir de datos de usuarios mediante algoritmos de machine learning.

\end{enumerate}
\end{solucionbox}

\newpage
% ═══════════════════════════════════════════════════════════════════════════
\appendix
\section{Apéndice: Resumen de Fórmulas y Resultados Clave}
% ═══════════════════════════════════════════════════════════════════════════

\begin{center}
\renewcommand{\arraystretch}{1.6}
\begin{tabular}{c c c}
\toprule
\textbf{Ángulo $\theta$} & \textbf{Similitud coseno} & \textbf{Interpretación} \\
\midrule
$0°$             & $1$     & Dirección idéntica (máxima similitud) \\
$0° < \theta < 90°$  & $0 < \cos\theta < 1$ & Parcialmente similares \\
$90°$            & $0$     & Ortogonales: sin relación \\
$90° < \theta < 180°$ & $-1 < \cos\theta < 0$ & Parcialmente opuestos \\
$180°$           & $-1$    & Dirección opuesta (máxima oposición) \\
\bottomrule
\end{tabular}
\end{center}

\bigskip

\begin{center}
\fbox{\parbox{0.9\textwidth}{\centering
\textbf{Relación fundamental:}
$$\cos(\theta) = \frac{\comp_{\vect{b}}(\vect{a})}{\|\vect{a}\|} = \frac{\vect{a}\cdot\vect{b}}{\|\vect{a}\|\;\|\vect{b}\|}$$
La similitud coseno es la versión \emph{normalizada} del componente escalar de la proyección.
}}
\end{center}

\end{document}
